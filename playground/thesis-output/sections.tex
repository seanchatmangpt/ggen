% ============================================================================
% SECTIONS - Generated from RDF Ontology
% ============================================================================


% Chapter 1: Introduction: From Synchronization to Information Theory
% Section: Motivation: The Synchronization Problem

\section{ Motivation: The Synchronization Problem }

In large-scale software systems, maintaining consistency between ontologies, templates, and generated code represents a critical information synchronization challenge. Traditional approaches treat synchronization as string manipulation. We propose treating ggen sync as an information-theoretic operation on hyperdimensional vector spaces.


% Chapter 1: Introduction: From Synchronization to Information Theory
% Section: Core Thesis

\section{ Core Thesis }

ggen sync implements a form of hyperdimensional information projection: mapping high-dimensional semantic spaces (RDF ontologies) to lower-dimensional projections (generated code) while preserving information-theoretic properties such as mutual information and conditional entropy.


% Chapter 1: Introduction: From Synchronization to Information Theory
% Section: Contributions

\section{ Contributions }

1) Formal model of code generation as information projection. 2) Proof that incremental sync preserves Shannon entropy bounds. 3) Algorithm for optimal dimension reduction in CONSTRUCT queries.


% Chapter 2: Mathematical Foundations: Shannon Entropy & Vector Spaces
% Section: Shannon Entropy and Information Content

\section{ Shannon Entropy and Information Content }

Let X be a discrete random variable with probability distribution P(x). Shannon entropy is defined as H(X) = -Σ P(x) log P(x). In our context, X represents possible RDF triple selections, and H(X) quantifies the information content of an ontology.


% Chapter 2: Mathematical Foundations: Shannon Entropy & Vector Spaces
% Section: Hyperdimensional Computing

\section{ Hyperdimensional Computing }

Hyperdimensional spaces of dimension d >> 1 enable semantic representations. Each RDF triple maps to a binary vector in {-1,+1}^d. Semantic relationships emerge through vector operations: similarity via dot product, binding via element-wise multiplication.


% Chapter 2: Mathematical Foundations: Shannon Entropy & Vector Spaces
% Section: Mutual Information and Correlation

\section{ Mutual Information and Correlation }

Mutual information I(X;Y) measures the information shared between variables X and Y. For ggen sync: I(Ontology; GeneratedCode) quantifies how much the ontology constrains code generation.


% Chapter 3: ggen Sync: Hyperdimensional Projection Mechanism
% Section: Full Sync: Orthogonal Projection

\section{ Full Sync: Orthogonal Projection }

Full sync mode regenerates all artifacts from source ontology. Formally, this is an orthogonal projection from the high-dimensional semantic space R^d (RDF triples) to lower-dimensional code space R^k where k << d.


% Chapter 3: ggen Sync: Hyperdimensional Projection Mechanism
% Section: Incremental Sync: Conditional Projection

\section{ Incremental Sync: Conditional Projection }

Incremental sync computes the projection conditioned on existing code: P(new code | existing code, ontology changes). This preserves manual edits marked with MANUAL pragma, implementing information retention.


% Chapter 3: ggen Sync: Hyperdimensional Projection Mechanism
% Section: Verify Mode: Information Consistency Check

\section{ Verify Mode: Information Consistency Check }

Verify mode computes the mutual information between source ontology and target code without modification. If I(O; Generated) < threshold, report conflicts.


% Chapter 4: SPARQL CONSTRUCT: Extracting Patterns from Ontology Space
% Section: CONSTRUCT as Semantic Compression

\section{ CONSTRUCT as Semantic Compression }

SPARQL CONSTRUCT queries implement lossy compression on RDF spaces. A CONSTRUCT query selects relevant triples (loss function) and reorders them (dimensionality reduction). The result is a smaller RDF graph encoding the essential information.


% Chapter 4: SPARQL CONSTRUCT: Extracting Patterns from Ontology Space
% Section: Template Rendering: Final Projection to Code

\section{ Template Rendering: Final Projection to Code }

Tera templates perform the final transformation: CONSTRUCT output → LaTeX/Rust/TypeScript. Template variables map to RDF triple properties. Template logic implements the code generation algorithm.


% Chapter 4: SPARQL CONSTRUCT: Extracting Patterns from Ontology Space
% Section: Fidelity: Measuring Information Loss

\section{ Fidelity: Measuring Information Loss }

Define fidelity F as the KL-divergence between original and reconstructed information spaces: F = D_KL(P(O) || P(Reconstruct(O))). Lower fidelity means information loss during projection.


% Chapter 5: ggen.toml as Configuration Space: Encoding Synchronization Strategy
% Section: Configuration as State Vector

\section{ Configuration as State Vector }

Each ggen.toml file represents a point in a configuration space Ω. The state vector s ∈ Ω encodes: sync mode, conflict resolution strategy, template selection, output formatting. The trajectory through Ω during multiple syncs traces the evolution of synchronization state.


% Chapter 5: ggen.toml as Configuration Space: Encoding Synchronization Strategy
% Section: Phase Space Dynamics

\section{ Phase Space Dynamics }

ggen sync traces a trajectory in phase space: (Ontology State, Code State, Configuration) evolves over time. Stable configurations correspond to attractors where I(Ontology; Code) is maximized subject to ggen.toml constraints.


% Chapter 5: ggen.toml as Configuration Space: Encoding Synchronization Strategy
% Section: Optimal Configuration Search

\section{ Optimal Configuration Search }

Finding optimal ggen.toml is equivalent to optimizing over Ω: max_{s ∈ Ω} I(O; C) - λ||s||_1. This is a constrained optimization problem solvable via gradient descent on the configuration space.


% Chapter 6: Conclusions: Towards Semantic Synchronization
% Section: Summary of Results

\section{ Summary of Results }

We have shown: (1) ggen sync is a legitimate information-theoretic operation. (2) CONSTRUCT queries extract maximum information while preserving fidelity bounds. (3) ggen.toml encodes optimal synchronization strategy. (4) Hyperdimensional encoding unifies code generation theory.


% Chapter 6: Conclusions: Towards Semantic Synchronization
% Section: Practical Implications

\section{ Practical Implications }

These theoretical results guide practical implementation: choose sync modes to preserve information bounds, validate CONSTRUCT queries for semantic completeness, design ggen.toml with entropy constraints in mind.


% Chapter 6: Conclusions: Towards Semantic Synchronization
% Section: Future Directions

\section{ Future Directions }

Extend to: (1) Multi-agent synchronization (game-theoretic analysis). (2) Quantum information analog. (3) Neural network approximation of CONSTRUCT queries. (4) Real-time adaptive ggen.toml based on entropy monitoring.


