\begin{figure}[h]
\centering
\begin{tikzpicture}[
    class/.style={rectangle, draw, fill=yellow!20, minimum width=2.5cm, minimum height=1cm, font=\small\bfseries},
    instance/.style={rectangle, draw, fill=orange!15, rounded corners, minimum width=2cm, minimum height=0.7cm, font=\small},
    property/.style={ellipse, draw, fill=cyan!15, minimum width=1.5cm, minimum height=0.6cm, font=\footnotesize},
    relation/.style={->, >=stealth, thick},
    inheritance/.style={->, >=stealth, thick, dashed},
    constraint/.style={rectangle, draw, fill=red!10, minimum width=1.8cm, minimum height=0.5cm, font=\tiny}
]

% Core classes
\node[class] (entity) at (0,4) {Entity};
\node[class] (service) at (4,4) {Service};
\node[class] (endpoint) at (0,2) {Endpoint};
\node[class] (parameter) at (-3,0) {Parameter};
\node[class] (response) at (0,0) {Response};
\node[class] (error) at (3,0) {Error};

% Properties
\node[property] (name) at (2,5) {name};
\node[property] (type) at (-2,3) {type};
\node[property] (required) at (-4.5,1) {required};

% Relationships
\draw[relation] (service) -- node[above, font=\tiny] {hasEndpoint} (endpoint);
\draw[relation] (endpoint) -- node[left, font=\tiny] {hasParameter} (parameter);
\draw[relation] (endpoint) -- node[right, font=\tiny] {hasResponse} (response);
\draw[relation] (endpoint) -- node[above, font=\tiny] {hasError} (error);
\draw[inheritance] (endpoint) -- node[right, font=\tiny] {inheritsFrom} (entity);

% Constraints
\node[constraint] (c1) at (5,2) {minLength: 1};
\node[constraint] (c2) at (5,1.3) {pattern: \^{}[a-z]+};
\draw[relation, color=red!60] (parameter) -- node[above, font=\tiny] {hasConstraint} (c1);

% Example instances
\node[instance] (blogpost) at (6,4.5) {BlogPost};
\node[instance, font=\tiny] (title) at (7.5,3.8) {title: string};
\node[instance, font=\tiny] (content) at (7.5,3.3) {content: string};
\node[instance, font=\tiny] (author) at (7.5,2.8) {author: Author};

\draw[relation, color=green!60] (blogpost) -- (title);
\draw[relation, color=green!60] (blogpost) -- (content);
\draw[relation, color=green!60] (blogpost) -- (author);

% Legend
\node[draw, rectangle, fill=white, minimum width=2.5cm, minimum height=2.2cm] at (8,-0.5) {};
\node[font=\tiny\bfseries] at (8,0.5) {Legend};
\node[class, minimum width=1.5cm, minimum height=0.4cm, font=\tiny] at (8,0) {Class};
\node[instance, minimum width=1.5cm, minimum height=0.3cm, font=\tiny] at (8,-0.4) {Instance};
\draw[relation] (7.2,-0.8) -- (7.8,-0.8);
\node[font=\tiny] at (8.5,-0.8) {Relation};
\draw[inheritance] (7.2,-1.1) -- (7.8,-1.1);
\node[font=\tiny] at (8.5,-1.1) {Inheritance};

\end{tikzpicture}
\caption{RDF Ontology Structure for API Contract Generation. The semantic model defines core classes (Entity, Service, Endpoint) and their relationships, enabling formal specification of API contracts with constraints and type information. Example instance shown for BlogPost entity.}
\label{fig:rdf-ontology}
\end{figure}
