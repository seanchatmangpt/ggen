\chapter{CLI Reference}
\label{app:cli-reference}

This appendix provides a comprehensive reference for the ggen wizard command-line interface.

\section{Wizard Commands}

\subsection{ggen wizard new}

Create a new project from a natural language description.

\begin{lstlisting}[language=bash]
ggen wizard new [OPTIONS] <DESCRIPTION>

ARGUMENTS:
  <DESCRIPTION>    Natural language project description

OPTIONS:
  -o, --output <DIR>       Output directory [default: .]
  -i, --interactive        Enable interactive mode
      --dry-run            Preview without generating
      --format <FORMAT>    Output format: human, json, yaml
  -v, --verbose            Verbose output
  -h, --help               Print help
\end{lstlisting}

\textbf{Example:}
\begin{lstlisting}[language=bash]
ggen wizard new "REST API for user management with JWT auth" \
    --output ./user-service \
    --interactive
\end{lstlisting}

\subsection{ggen wizard add}

Add a feature to an existing project specification.

\begin{lstlisting}[language=bash]
ggen wizard add [OPTIONS] <FEATURE>

ARGUMENTS:
  <FEATURE>    Natural language feature description

OPTIONS:
  -p, --project <PATH>     Project path [default: .]
      --dry-run            Preview without modifying
  -h, --help               Print help
\end{lstlisting}

\subsection{ggen wizard modify}

Modify existing specifications conversationally.

\begin{lstlisting}[language=bash]
ggen wizard modify [OPTIONS] <CHANGE>

ARGUMENTS:
  <CHANGE>    Natural language change description

OPTIONS:
  -p, --project <PATH>     Project path [default: .]
      --dry-run            Preview without modifying
  -h, --help               Print help
\end{lstlisting}

\subsection{ggen wizard explain}

Generate human-readable explanation of RDF specifications.

\begin{lstlisting}[language=bash]
ggen wizard explain [OPTIONS] <SPEC_FILE>

ARGUMENTS:
  <SPEC_FILE>    Path to .ttl specification file

OPTIONS:
      --format <FORMAT>    Output format: human, json, markdown
      --diagrams           Include entity relationship diagrams
  -h, --help               Print help
\end{lstlisting}

\section{Pack Commands}

\subsection{Discovery Commands}

\begin{lstlisting}[language=bash]
ggen pack list [OPTIONS]
ggen pack search <QUERY> [OPTIONS]
ggen pack show <PACK_NAME> [OPTIONS]
ggen pack info <PACK_NAME>
\end{lstlisting}

\subsection{Management Commands}

\begin{lstlisting}[language=bash]
ggen pack install <PACK_NAME> [OPTIONS]
ggen pack uninstall <PACK_NAME> [OPTIONS]
ggen pack update [PACK_NAME] [OPTIONS]
ggen pack clean [OPTIONS]
\end{lstlisting}

\subsection{Generation Commands}

\begin{lstlisting}[language=bash]
ggen pack generate <PACK_NAME> --output <DIR> [OPTIONS]
ggen pack regenerate [OPTIONS]
\end{lstlisting}

\subsection{Composition Commands}

\begin{lstlisting}[language=bash]
ggen pack compose --packs <P1,P2,...> --output <DIR> [OPTIONS]
ggen pack merge <SOURCE_DIR> <TARGET_DIR> [OPTIONS]
ggen pack plan --packs <P1,P2,...> [OPTIONS]
\end{lstlisting}

\subsection{Validation Commands}

\begin{lstlisting}[language=bash]
ggen pack validate <PACK_NAME> [OPTIONS]
ggen pack lint <PACK_PATH> [OPTIONS]
ggen pack check <PACK_NAME>
\end{lstlisting}

\subsection{Publishing Commands}

\begin{lstlisting}[language=bash]
ggen pack publish <PACK_PATH> [OPTIONS]
ggen pack create [OPTIONS]
ggen pack init [OPTIONS]
\end{lstlisting}

\subsection{Scoring Commands}

\begin{lstlisting}[language=bash]
ggen pack benchmark <PACK_NAME> [OPTIONS]
ggen pack score <PACK_NAME> [OPTIONS]
\end{lstlisting}

\subsection{Utility Commands}

\begin{lstlisting}[language=bash]
ggen pack tree <PACK_NAME> [OPTIONS]
ggen pack diff <PACK_NAME> <V1> <V2> [OPTIONS]
ggen pack export <PACK_NAME> [OPTIONS]
ggen pack import <FILE> [OPTIONS]
\end{lstlisting}

\section{Core Commands}

\subsection{ggen sync}

Synchronize generated code from ontology.

\begin{lstlisting}[language=bash]
ggen sync [OPTIONS]

OPTIONS:
      --dry-run       Preview without generating
      --force         Force overwrite existing files
      --audit         Generate audit trail
  -h, --help          Print help
\end{lstlisting}

\subsection{ggen validate}

Validate project configuration and specifications.

\begin{lstlisting}[language=bash]
ggen validate [OPTIONS]

OPTIONS:
      --closure-proof    Verify specification closure
      --strict           Fail on warnings
  -h, --help             Print help
\end{lstlisting}

\section{Exit Codes}

\begin{table}[h]
\centering
\begin{tabular}{cl}
\toprule
\textbf{Code} & \textbf{Meaning} \\
\midrule
0 & Success \\
1 & General error \\
2 & Validation failure \\
3 & Specification not closed \\
4 & Template error \\
5 & Network error \\
\bottomrule
\end{tabular}
\caption{ggen CLI exit codes}
\end{table}

\section{Environment Variables}

\begin{table}[h]
\centering
\begin{tabular}{lp{8cm}}
\toprule
\textbf{Variable} & \textbf{Description} \\
\midrule
\texttt{GGEN\_HOME} & ggen home directory (default: \texttt{\textasciitilde/.ggen}) \\
\texttt{OPENAI\_API\_KEY} & OpenAI API key for wizard AI features \\
\texttt{ANTHROPIC\_API\_KEY} & Anthropic API key for Claude integration \\
\texttt{GGEN\_LOG\_LEVEL} & Log level (trace, debug, info, warn, error) \\
\texttt{GGEN\_NO\_COLOR} & Disable colored output \\
\bottomrule
\end{tabular}
\caption{ggen environment variables}
\end{table}
