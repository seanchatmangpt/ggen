% ============================================================================
% CHAPTER 5: SPECIFICATION-FIRST CODE GENERATION FRAMEWORK
% ============================================================================

\chapter{Specification-First Code Generation Framework}
\label{ch:framework}

\section{Overview}

This chapter describes the complete framework for specification-first code generation, implementing the Chatman Equation $A = \mu(O)$ in practice.

\section{Architecture}

\subsection{Layered Architecture}

The framework consists of four layers:

\begin{equation}
\begin{array}{|c|}
\hline
\text{Generated Code (JavaScript/TypeScript)} \\
\hline
\text{Emission Layer (Tera Templates)} \\
\hline
\text{Extraction Layer (SPARQL Queries)} \\
\hline
\text{Specification Layer (RDF/Turtle + SHACL)} \\
\hline
\end{array}
\end{equation}

\subsection{Five-Stage Pipeline}

The complete generation process follows five stages:

\begin{figure}[H]
\centering
\begin{tikzpicture}[scale=0.8]
  \node[draw, rectangle, minimum width=2cm, minimum height=1cm] (stage1) at (0,0) {Stage 1\\Normalize};
  \node[draw, rectangle, minimum width=2cm, minimum height=1cm] (stage2) at (3,0) {Stage 2\\Extract};
  \node[draw, rectangle, minimum width=2cm, minimum height=1cm] (stage3) at (6,0) {Stage 3\\Emit};
  \node[draw, rectangle, minimum width=2cm, minimum height=1cm] (stage4) at (9,0) {Stage 4\\Canonicalize};
  \node[draw, rectangle, minimum width=2cm, minimum height=1cm] (stage5) at (12,0) {Stage 5\\Receipt};

  \draw[->,thick] (stage1) -- (stage2);
  \draw[->,thick] (stage2) -- (stage3);
  \draw[->,thick] (stage3) -- (stage4);
  \draw[->,thick] (stage4) -- (stage5);

  \node[below=1cm of stage1, font=\small] {RDF/Turtle};
  \node[below=1cm of stage2, font=\small] {Data};
  \node[below=1cm of stage3, font=\small] {Code};
  \node[below=1cm of stage4, font=\small] {Formatted};
  \node[below=1cm of stage5, font=\small] {Verified};
\end{tikzpicture}
\caption{Five-stage code generation pipeline}
\label{fig:pipeline}
\end{figure}

\section{Stage 1: Normalization}

\subsection{Input}

Raw RDF files in Turtle format with:
- Undefined namespace prefixes
- Relative file paths
- Inconsistent naming conventions

\subsection{Processing}

\begin{enumerate}
    \item \textbf{Prefix Expansion}: Convert \texttt{@prefix} declarations to full IRIs
    \item \textbf{URI Validation}: Verify all IRIs are well-formed
    \item \textbf{File Merging}: Combine multiple Turtle files
    \item \textbf{Syntax Checking}: Verify Turtle syntax correctness
\end{enumerate}

\subsection{Output}

Canonical RDF graph with:
- All prefixes expanded
- All URIs validated
- Consistent representation

\section{Stage 2: Extraction}

\subsection{Input}

Canonical RDF graph from Stage 1.

\subsection{Processing}

Execute SPARQL CONSTRUCT queries to transform RDF:

\begin{lstlisting}[language=SQL]
CONSTRUCT {
  ?job a bree:CompiledJobConfig ;
    bree:configJson ?configJson ;
    bree:jobName ?name ;
    bree:jobPath ?path .
}
WHERE {
  ?job a bree:Job ;
    bree:jobName ?name ;
    bree:jobPath ?path .
}
\end{lstlisting}

\subsection{Output}

Extracted data in structured format (typically JSON or YAML), ready for template substitution.

\section{Stage 3: Emission}

\subsection{Input}

Extracted data from Stage 2.

\subsection{Processing}

Apply Tera templates to generate code:

\begin{lstlisting}[language=jinja]
const bree = new Bree({
  root: path.join(__dirname, '{{ instance.root }}'),
  removeCompleted: {{ instance.removeCompleted|lower }},
  jobs: [
    
    {
      name: '{{ job.jobName }}',
      path: path.join(__dirname, '{{ job.jobPath }}'),
      interval: {{ job.interval }},
    },
    
  ],
});
\end{lstlisting}

\subsection{Output}

Generated JavaScript/TypeScript code.

\section{Stage 4: Canonicalization}

\subsection{Input}

Raw generated code from Stage 3.

\subsection{Processing}

\begin{enumerate}
    \item \textbf{Formatting}: Apply code style (Prettier, eslint)
    \item \textbf{Syntax Validation}: Run parser to verify correctness
    \item \textbf{Type Checking}: TypeScript type checking (if applicable)
    \item \textbf{Linting}: Code quality analysis
\end{enumerate}

\subsection{Output}

Production-quality code meeting all quality standards.

\section{Stage 5: Receipt}

\subsection{Input}

Canonical code from Stage 4.

\subsection{Processing}

\begin{enumerate}
    \item \textbf{Checksumming}: Compute blake3 hash of output
    \item \textbf{Manifest Generation}: Create list of all generated files
    \item \textbf{Verification}: Compare hashes with previous generation (if available)
    \item \textbf{Logging}: Record generation metadata
\end{enumerate}

\subsection{Output}

Receipt containing:

\begin{lstlisting}[language=json]
{
  "specification_hash": "blake3:abc123...",
  "generated_code_hash": "blake3:def456...",
  "timestamp": "2026-01-07T18:00:00Z",
  "files": [
    { "path": "generated/bree-instance.js", "hash": "..." },
    { "path": "generated/citty-cli-main.js", "hash": "..." }
  ],
  "deterministic": true
}
\end{lstlisting}

\section{Big Bang 80/20: Specification Validation}

Before entering the generation pipeline, specifications must achieve \textit{closure}:

\begin{equation}
\text{Closure} = \text{All required properties present} \land \text{No conflicts}
\end{equation}

\subsection{SHACL Validation}

We use SHACL shapes to validate:

\begin{lstlisting}[language=turtle]
bree:JobShape
  a sh:NodeShape ;
  sh:targetClass bree:Job ;

  sh:property [
    sh:path bree:jobName ;
    sh:datatype xsd:string ;
    sh:minCount 1 ;
    sh:maxCount 1 ;
    sh:message "Job must have exactly one jobName" ;
  ] ;

  sh:sparql [
    a sh:SPARQLConstraint ;
    sh:message "Job must have timing specification" ;
    sh:select """
      SELECT $this
      WHERE {
        FILTER NOT EXISTS { $this (bree:hasTimeout | bree:hasInterval | bree:hasCron | bree:hasDate) ?timing . }
      }
    """ ;
  ] .
\end{lstlisting}

\subsection{Closure Algorithm}

\begin{algorithm}
\caption{Specification Closure Validation}
\label{alg:closure}
\begin{algorithmic}
\Require Specification RDF graph $O$, SHACL shapes $\Sigma$
\Ensure $O$ achieves closure or error

\For{each shape $\sigma \in \Sigma$}
  \For{each instance $i$ matching $\sigma$}
    \For{each property constraint $(p, C) \in \sigma$}
      \If{not satisfies($i$, $p$, $C$)}
        \State \Return \texttt{Error: Constraint violation}
      \EndIf
    \EndFor
  \EndFor
\EndFor

\State \Return \texttt{Success: Specification achieves closure}
\end{algorithmic}
\end{algorithm}

\section{Determinism Guarantee}

\subsection{Theorem: Pipeline Determinism}

\begin{theorem}
If each stage $S_i$ is deterministic (pure function, no side effects), then the complete pipeline is deterministic:
\begin{equation}
\forall O: \text{Pipeline}(O) = \text{Pipeline}(O)
\end{equation}
\end{theorem}

\begin{proof}
By induction:
\begin{enumerate}
    \item Base case: Stage 1 is deterministic (pure function of input RDF)
    \item Inductive step: If Stage $i$ is deterministic, Stage $i+1$ input is deterministic
    \item Therefore: Stage $i+1$ output is deterministic (composition of pure functions)
    \item Conclusion: All five stages are deterministic, so pipeline is deterministic
\end{enumerate}
\end{proof}

\subsection{Hash-Based Verification}

Determinism enables hash-based verification:

\begin{equation}
\text{blake3}(O) \xrightarrow{\text{deterministic}} A \xrightarrow{\text{hash}} \text{blake3}(A)
\end{equation}

Two deployments with:
- Same specification hash $\text{blake3}(O)$
- Always produce identical code

This enables:
- \textbf{Auditability}: Code commit hash uniquely identifies specification
- \textbf{Reproducibility}: Any developer can regenerate identical code
- \textbf{Compliance}: Specification → Code mapping is traceable

\section{Configuration: ggen-bree-config.toml}

\subsection{Structure}

\begin{lstlisting}[language=toml]
[specification]
description = "Bree Job Scheduler Code Generation"
rdf_source = "bree-ontology.ttl"
rdf_jobs_data = "bree-jobs-sample.ttl"
shacl_shapes = ".specify/bree-scheduler.shapes.ttl"
output_dir = "./generated"

[[entities]]
name = "BreeInstance"
sparql_class = "bree:BreeInstance"
sparql_query = "SELECT ?instance ?name ... WHERE { ... }"

[entities.generation.javascript]
template = "templates/bree-instance.js.tera"
output_file = "generated/bree-instance.js"

[[transformations]]
name = "JobToCittyCommand"
sparql_construct = "CONSTRUCT { ... } WHERE { ... }"

[transformations.generation]
template = "templates/citty-command.js.tera"

[[validations]]
name = "BreeInstanceShape"
shapes_file = ".specify/bree-scheduler.shapes.ttl"
target_class = "bree:BreeInstance"

[pipeline]
normalization = { expand_prefixes = true, validate_uris = true }
extraction = { construct_patterns = ["bree-construct-patterns.sparql"] }
emission = { template_engine = "tera", parallel_rendering = true }
canonicalization = { format_javascript = true }
receipt = { verify_javascript = true, generate_checksums = true }
\end{lstlisting}

\section{Templates: Tera Integration}

\subsection{Template Variables}

Templates receive extracted data:

\begin{lstlisting}[language=jinja]

  const job_{{ loop.index }} = {
    name: "{{ job.jobName }}",
    path: "{{ job.jobPath }}",
    timeout: {{ job.closeWorkerAfterMs }},
  };

\end{lstlisting}

\subsection{Conditional Rendering}

\begin{lstlisting}[language=jinja]

  cron: "{{ job.cron }}",

  interval: {{ job.interval }},

  timeout: {{ job.timeout }},

\end{lstlisting}

\subsection{Filters and Functions}

\begin{lstlisting}[language=jinja]
timeout: {{ timeout|default(value=30000) }},
removeCompleted: {{ removeCompleted|lower }},
jobCount: {{ jobs|length }},
\end{lstlisting}

\section{Summary}

The specification-first code generation framework:

\begin{enumerate}
    \item Validates specifications achieve closure (SHACL)
    \item Normalizes and canonicalizes RDF
    \item Extracts data via SPARQL queries
    \item Generates code via Tera templates
    \item Validates output code quality
    \item Produces cryptographic receipt for auditability
\end{enumerate}

This five-stage pipeline guarantees determinism and auditability at enterprise scale.

\newpage
